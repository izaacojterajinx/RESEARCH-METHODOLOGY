\documentclass[12pt]{article}

\usepackage{graphicx}
\begin{document}

\begin{center}
\begin{normalsize}
\textbf\sl{MAKERERE \includegraphics[scale=0.5]{logo} UNIVERSITY }\\

\textbf\sl{SCHOOL OF COMPUTING AND INFORMATICS TECHNOLOGY} \\
\textbf\sl{DEPARTMENT OF COMPUTER SCIENCE} \\
\textbf\sl{BIT 2207 RESEARCH METHODOLOGY} \\
\textbf\sl{LECTURER: ERNEST MWEBAZE} \\
\paragraph*{•}
\textbf{Prepared by:}\\
\textbf{\sc OJOK ISAAC } \\
\textbf{\sc Reg No: 16/U/20048/PS } \\
\textbf{\sc std No: 216021703}\\
\paragraph*{•}
\
\newpage


\textsc{ LITERATURE ABOUT GOOGLE TRANSLATE}\\ 

\end{normalsize}
\end{center}

\section{Introduction}
\paragraph{•}
Google translate is a free multilingual machine translation service developed by google ,to translate text.It offers a website interface,mobile apps for andriod and ios,and an API that helps developers to build extensions and software applications.
Google translate supports over 100 languages at various levels and as of May 2017, it was serving over 500 million people daily.Google maps was launched in April 28,2006(as statistical machine translation)[1] then on November 15,2016(as neural machine translation)[2].
google translate can translate various forms of text and media including text,speech,images,sites or real time video ,from one language to another.[3][4]For some languages google translate can pronouce some translated text,[5]highlight corresponding words and phrases in the course and target text,and act as a simple dictionary for single-word input.


\section{literature review}
\paragraph{•}
[6]The current study was designed to form a collaboration of EPCs to better analyze the accuracy of the freely available, online, translation tool—Google Translate—for the purposes of data extraction of articles in selected non-English languages. The collaboration allowed for double data extraction and a better consensus determination of the important extraction items to assess; we also implemented an improved analytic technique.

The research had the following aims:
Compare data extraction of trials done on original-language articles by native speakers with data extraction done on articles translated to English by Google Translate.
Track and enumerate the time and resources used for article translation and the extra time and resources required for data extraction related to use of translated articles.

Users do claim that the app has many languages it supports hence making it easy to message and communicate and in the long run many have learnt new languages because of using it.
The users of google translate claim that the app is of great help to them especially whenm they travel to places where they have no clue of how to communicate to the people in that given area at that particular time.
Users also claim that the app at times translates the texts in a different meaning hence leading to mis communication.

\paragraph{•}




\begin{thebibliography}{9}
\bibitem{}
https://en.wikipedia.org/wiki/Google-translate
\bibitem{}
http://blog.google/products/translate/found-translate-more-accurate-fluent-sentence-google-translate
 \bibitem{}
https://blog.google/products/translate/found-translate-more-accurate-fluent-sentence-google-translate
 \bibitem{}
https://support.google.com/translate/topic=7011659
\bibitem{}
https://translate.google.com/intl/en/about/languages/
\bibitem{}
https://www.ncbi.nlm.nih.gov/books/NBK121301/



\end{thebibliography}


\end{document}
