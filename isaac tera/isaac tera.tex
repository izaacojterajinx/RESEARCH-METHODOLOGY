\documentclass[a4paper,12pt]{article}
\begin{document}


\begin{Huge}
\begin{center}
\begin{normalsize}

\textbf{MAKERERE UNIVERSITY } \\
\textbf{BACHELOR OF SCIENCE IN COMPUTER SCIENCE} \\
\textbf{BIT 2207 RESEARCH METHODOLOGY} \\
\textbf{YEAR 2} \\ 



\textbf{\sc OJOK ISAAC } \\
\textbf{\sc Reg No: 16/U/20048/PS } \\
\textbf{\sc std No: 216021703}\\
\end{normalsize}
\end{center}
\end{Huge}
\newpage

\begin{Huge}
\begin{center}
\begin{normalsize}
\textbf{\  SAFE  CLEAN DRINKING WATER FOR TESO REGION}
\end{normalsize}
\end{center}
\end{Huge}
\section{\  Introduction}
Safe drinking water is the water that is meant for human drinking and it is free from germs,bacteria and its clean and safely stored.
\paragraph{\sl  Majority of the families living in africa as a continet have found difficultty in finding and attaining safe clean drinking water on a daily basis,and regardless of the various numerous water bodies within theregion  still there is insufficiency of safe clean drinking water to the available families.}
During the dry seasons,many of the water sources dry out leaving the available water sources to suffocate and the popoulation at large suffering due to the reduction in the amount of safe clean drinking water.
In this report we took a further look at the various number of water sources and the number of the population in the region to know how best to avail safe drinking water to the families in the region


\section{\  Research Methods}
\paragraph{ \sl Using quantitative research methods,we issued out questionnaires to people in the towns within the region.}
\paragraph{\sl  The questionnaires were to attain information about the amount of drinking water each family consumes per aday and weekly and we sampled about 450 families.Personal information was collected to be used for future reference incase when needed in future .}
Below is a table attached showing the sample of the data collected.
\begin{center}
\begin{tabular}{|c|c|c|c|}
\hline
District & Source of water(majority) & Clean safe H2O& water per day (litres) \\ [0.5ex]
\hline
SOROTI & Tap  & YES & 64\\ [0.5ex]
\hline
KOTIDO & Springs & NO & 66\\ [0.5ex]
\hline
KABERAMAIDO & wells,taps,springs & NO & 178\\ [0.5ex]
\hline
KATAKWI & Tap,springs & moderate & 97 \\ [0.5ex]
\hline
KUMI & Tap& YES & 45 \\ [0.5ex]
\hline
\end{tabular}
\end{center}
Using analytical research technique,the research can be seen that the information results in the table show that their are few families that have access to clean safe drinking water(26\%) compared to the 74\% of the families that were surveyed have no access to safe clean water.
\section{\  Conclusion}
\paragraph{\sl The insufficient availability of safe clean drinking water to the locals in the region of teso is clearly indicated in the research and there is need to combat it immediately.
.}



\section{\   Recommendation}
It is recommended that teso families obtain clean storage places to store the water after them collecting it from the various water sources inorder for it not to be contaminated.We do recommend that more water sources i.e. boreholes be constructed to supplement the already available ones,so as not to suffocate the families during the dry season.Boiling of the collected water has to be considered greatly amongst the locals.And finally the water sources need to be maintained and kept clean at all the times.



\end{document}